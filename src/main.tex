\documentclass{article}

\usepackage[utf8]{inputenc}
\usepackage[margin=1in]{geometry}
% math symbols etc
\usepackage{amsmath,amsthm,amssymb}
% declarepaireddelimiter for \norm
\usepackage{mathtools}


% \degree symbol
\usepackage{gensymb}
% verbatim (code blocks)
\usepackage{verbatim}
% titlesec to redefine \section to get number after title
\usepackage[explicit]{titlesec}
% \textcolor names
\usepackage[usenames, dvipsnames]{color}
\usepackage{xcolor}
\usepackage{tikz}
\usepackage{lmodern}
\usepackage[T1]{fontenc}
\usepackage[finnish]{babel}

\usetikzlibrary{automata,positioning}

\setlength{\parindent}{0pt}
 
\newcommand{\N}{\mathbb{N}}
\newcommand{\Z}{\mathbb{Z}}
\newcommand{\R}{\mathbb{R}}
\newcommand{\Q}{\mathbb{Q}}
\newcommand{\mapsTo}{\rightarrow}


\newcommand{\cfbox}[2]{%
    \colorlet{currentcolor}{.}%
    {\color{#1}%
    \fbox{\color{currentcolor}#2}}%
}
\newcommand{\Lim}[1]{\raisebox{0.5ex}{\scalebox{0.8}{$\displaystyle \lim_{#1}\;$}}}
\newcommand\eqover[1]{\stackrel{\mathclap{\small\mbox{#1}}}{=}}

\DeclarePairedDelimiter\norm{\lVert}{\rVert}%

\title{Example Document}
\date{18.1.2021}
% number after title
\titleformat{\section}{\normalfont\Large\bfseries}{}{0em}{#1\ \thesection}
% subsection has letter, e.g. 1.1 is 1.a
\renewcommand{\thesubsection}{\thesection.\alph{subsection}}
\renewcommand{\thesubsubsection}{\thesubsection.\roman{subsubsection}}

\begin{document}
\maketitle

\section{Tehtävä}


\begin{align*}
    \frac{\cos(x + h) - \cos(x)}{h} &= \frac{\cos(x)\cos(h) - \sin(x)\sin(h) - \sin(x)}{h}\\
    &= \frac{\cos(x)(\cos(h) - 1) - \sin(x)\sin(h)}{h}\\
    &= \cos(x) \Lim{h \rightarrow 0} \frac{\cos(h) - 1}{h} - \sin(x) \Lim{h \rightarrow 0} \frac{\sin(h)}{h}\\
    &= \cos(x) \cdot 0 - \sin(x) \cdot 1\\
    &= -\sin(x)
\end{align*}

\section{Tehtävä}
\subsection{}

\begin{align*}
    D_x 3 + \sin \sqrt[3]{x^2} &= D_x 3 + D_x \sin \sqrt[3]{x^2}\\
    &= D_x \sin u
\end{align*}

\begin{align*}
    u &= \sqrt[3]{u'}\\
    u' &= x^2\\
    \frac{d}{du}(\sqrt[3]{u'}) &= \frac{1}{3 \sqrt[3]{u'^2}}\\
    \text{Ketjusäännöllä saadaan:}\\
    \frac{d\sqrt[3]{u'}}{du'} \frac{du'}{dx} &= \frac{2x}{3 \sqrt[3]{x^4}}\\
    \text{Ketjusäännöllä saadaan:}\\
    \frac{d \sin u}{du} \frac{du}{dx} &= \frac{2x \cos \sqrt[3]{x^2}}{3 \sqrt[3]{x^4}}\\
    \text{eli:}\\
    D_x 3 + \sin \sqrt[3]{x^2} &= \frac{2x \cos \sqrt[3]{x^2}}{3 \sqrt[3]{x^4}}
\end{align*}

\subsection{}

\begin{align*}
    D_x \sqrt{1 - 2x + x^2} &= D_x \sqrt{u}\\
    u &= 1 - 2x - x^2
\end{align*}


\begin{align*}
    \frac{du}{dx} &= (1 - 2x - x^2) = 2x - 2\\
    \text{Ketjusäännöllä saadaan:}\\
    \frac{d \sqrt{u}}{du} \frac{du}{dx} &= \frac{1}{2 \sqrt{1 - 2x + x^2}} 2x - 2\\
    &= \frac{x - 1}{\sqrt{(x - 1)^2}}\\
    \text{eli:}\\
    D_x \sqrt{1 - 2x + x^2} &= \frac{x - 1}{\sqrt{(x - 1)^2}}\\
\end{align*}

\end{document}
