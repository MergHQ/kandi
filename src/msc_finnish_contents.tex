\chapter{Johdanto}

Seuraavassa on joitain ohjeita tämän tutkielmapohjan käyttöön maisterintutkielmassa. Kirjoittamisohjeita löytyy useasta eri lähteestä. Voit esimerkiksi tutustua kandidaatintutkielman ohjeisiin. 
Ohjaajan kanssa on hyvä keskustella aikaisessa vaiheessa työn rakenteesta.

\chapter{Kuvat ja Taulukot}

\section{Kuvat}
Kuva~\ref{fig:logo} toimii esimerkkinä kuvan lisäämisestä työhön. Muista myös viitata jokaiseen kuvaan tekstissä. 

\begin{figure}[h!] % remove [h!] for automatic placement, which is probalby better for a thesis with more text on page
\centering 
\includegraphics[width=0.3\textwidth]{HY-logo-ml.png}
\caption{Helsingin yliopiston logo matemaattis-luonnontieteellisen tiedekunnan värein.\label{fig:logo}}
\end{figure}

\section{Taulukot}

Taulukossa~\ref{table:results} on esimerkki kokeellisten tulosten raportoinnista taulukkona. Muista myös viitata jokaiseen taulukkoon tekstissä.

\begin{table}[h!]
\centering
\caption{Kokeelliset tulokset.\label{table:results}}
\begin{tabular}{l||l c r} 
Koe & 1 & 2 & 3 \\ 
\hline \hline 
$A$ & 2.5 & 4.7 & -11 \\
$B$ & 8.0 & -3.7 & 12.6 \\
$A+B$ & 10.5 & 1.0 & 1.6 \\
\hline
%
\end{tabular}
\end{table}

\chapter{Viitteet}

\section{Kirjallisuusviitteet}

Kirjallisuuslähteet ylläpidetään erillisessä .bib-tiedostossa. Tässä tutkielmapohjassa käytetyt kirjallisuuslähteet, joista esimerkkejä kuvassa~\ref{bibexamples}, löytyvät tiedostosta\newline \texttt{bibliography.bib}.

\begin{figure}[h!]
    \centering
    \begin{scriptsize}
				
\begin{verbatim}

@article{einstein,
    author =       "Albert Einstein",
    title =        "{Zur Elektrodynamik bewegter K{\"o}rper}. ({German})
        [{On} the electrodynamics of moving bodies]",
    journal =      "Annalen der Physik",
    volume =       "322",
    number =       "10",
    pages =        "891--921",
    year =         "1905",
    DOI =          "http://dx.doi.org/10.1002/andp.19053221004"
}
@book{latexcompanion,
    author    = "Michel Goossens and Frank Mittelbach and Alexander Samarin",
    title     = "The \LaTeX\ Companion",
    year      = "1993",
    publisher = "Addison-Wesley",
    address   = "Reading, Massachusetts"
}
@book{knuth99,
    author    = "Donald E. Knuth",
    title     = "Digital Typography",
    year      = "1999",
    publisher = "The Center for the Study of Language and Information",
    series    = "CLSI Lecture Notes (78)"
}
\end{verbatim}
\end{scriptsize}
    \caption{Esimerkkejä kirjallisuuslähteiden kuvaamisesta .bib-tiedostossa.}
    \label{bibexamples}
\end{figure}

Viitteet kirjallisuuslähteisiin muodostetaan komennolla \texttt{\textbackslash citep\{einstein\}}, josta generoituu tekstiin valitun viittaustyylin mukaisesti muotoiltu viite \citep{einstein}, tai \texttt{\textbackslash citep\{latexcompanion,knuth99\}}, josta tekstiin puolestaan generoituu \citep{latexcompanion,knuth99}. Tekstissä viitatut kirjallisuuslähteet tulevat automaattisesti viiteluetteloon. Kirjallisuuslähteiden tietojen oikeellisuus ja yhdenmukaisuus .bib-tiedostossa vaikuttavat luonnollisesti siihen, miten tiedot tutkielmassa näyttäytyvät. Tämä on syytä huomioida, sillä esim. verkosta valmiiksi {Bib\TeX} muodossa löytyvien tietojen täydellisyyten tai samanmuotoisuuteen ei pidä sokeasti luottaa.  

Keskustele viittaustyylin valinnasta ohjaajan kanssa. Joitain vaihtoehtoja on osoitteessa:\\ 
\url{https://www.overleaf.com/learn/latex/Biblatex_bibliography_styles}.
%\url{https://www.sharelatex.com/learn/Bibtex_bibliography_styles}.

\section{Ristiviitteet}

%Liite~\ref{appendix:model} sivulla~\pageref{appendix:model} sisältää lisämateriaalia.
Taulukossa~\ref{table:results} sivulla~\pageref{table:results} on koottuna kokeelliset tulokset.

% \chapter{Pdf:n luominen tex:stä}

% Linuxissa voit ajaa komentoja \texttt{pdflatex filename.tex} ja \texttt{biber filename.tex} vuorotelleen kunnes ohjelmat eivät enää anna varoituksia. Prosessi on helppo automatisoida make-komennon avulla.
 
% \chapter{Johtopäätökset\label{chapter:conclusions}}

% Tutkielma on hyvä päättää johtopäätöksiin tutkielman löydöksistä. 
