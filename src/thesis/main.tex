%% History:
%% December 2020 Veli Mäkinen removed obsolete options related to 40 cr theses
%% May 2019 Tomi Männistö, Antti-Pekka Tuovinen proofreading; 30 vs. 40 cr theses, etc.
%% May 2019 Tomi Männistö changed from babelbib to bibtex; Abstract page (and other pages as well) reformatting.
%% January–May 2019 several issues fixed by Niko Mäkitalo; long fields in abstract
%% March 2018 template file extended by Lea Kutvonen to exploit HYthesisML.cls.
%% Feb2018 This template file for the use of HYgraduML.cls was  modified by Veli Mäkinen from HY_fysiikka_LuKtemplate.tex
%% authored by Roope Halonen ja Tomi Vainio in 2017.
%% Some text is also inherited from engl_malli.tex versions by Kutvonen, Erkiö, Mäkelä, Verkamo, Kurhila, and
%% Nykänen, to accompany tktltiki.cls (by Puolakka 2002).


%% Follow comments to support use.

%%%%%%%%%%%%%%%%%%%%%%%%%%%%%%%%%%%%%%%%%%%%%%%%%%%%%%%%%
%% STEP 1: Choose options for MSc / BSc / seminar layout and your bibliographic style
%%%%%%%%%%%%%%%%%%%%%%%%%%%%%%%%%%%%%%%%%%%%%%%%%%%%%%%%%

%%  Language: 
%%      finnish, swedish, or english
%%  Pagination (use twoside by default)  
%%      oneside or twoside,
%%  Study programme / kind of report
%%      csm  = Master's thesis in Computer Science Master's Programme;
%%      tkt = Bachelor's thesis in Computer Science Bachelor's Programme;
%%      seminar = seminar report
%%  For Master's thesis choose your line or track:
%%      (30 cr thesis, 2020 onwards, Master's Programme in Computer Science = csm)
%%      software-track-2020 = Software study track
%%      algorithms-track-2020 = Algorithms study track
%%      networking-track-2020 = Networking study track
%%
%%      (30 cr thesis, Master's Programme in Computer Science = csm)
%%      sw-track-2018 = Software Systems study track
%%      alko-track-2018 = Algorithms study track
%%      nodes-track-2018 = Networking and Services study track
%%
%%      (30 cr thesis, Master's Programme in Computer Science = csm)
%%      sw-line-2017 =  Software systems subprogramme
%%      alko-line-2017 = Algorithms, Data Analytics and Machine Learning subprogramme
%%      bio-line-2017 = Algorithmic Bioinformatics subprogramme
%%      nodes-line-2017 = Networking and Services subprogramme
%%

\documentclass[swedish,twoside,censored,tkt]{HYthesisML}


% In theses, open new chapters only at right page.
% For other types of documents, may ask "openany" in document.
\PassOptionsToClass{openright,twoside,a4paper}{report}
%\PassOptionsToClass{openany,twoside,a4paper}{report}

\usepackage{csquotes}
%%%%%%%%%%%%%%%%%%%%%%%%%%%%%%%%%%%%%%%%%%%%%%%%%%%%%%%%%
%% REFERENCES
%% Some notes on bibliography usage and options:
%% natbib -> you can use, e.g., \citep{} or \parencite{} for (Einstein, 1905); with APA \cite -> Einstein, 1905 without ()
%% maxcitenames=2 -> only 2 author names in text citations, if more -> et al. is used
%% maxbibnames=99 as no great need to suppress the biliography list in a thesis
%% for more information see biblatex package documentation, e.g., from https://ctan.org/pkg/biblatex 

%% Reference style: select one 
%% for APA = Harvard style = authoryear -> (Einstein, 1905) use:
%\usepackage[style=authoryear,bibstyle=authoryear,backend=biber,natbib=true,maxnames=99,maxcitenames=2,giveninits=true,uniquename=init]{biblatex}
%% for numeric = Vancouver style -> [1] use:
\usepackage[style=numeric,bibstyle=numeric,backend=biber,natbib=true,maxbibnames=99,giveninits=true,uniquename=init]{biblatex}
%% for alpahbetic -> [Ein05] use:
%\usepackage[style=alphabetic,bibstyle=alphabetic,backend=biber,natbib=true,maxbibnames=99,giveninits=true,uniquename=init]{biblatex}
%

\addbibresource{bibliography.bib}
% in case you want the final delimiter between authors & -> (Einstein & Zweistein, 1905) 
% \renewcommand{\finalnamedelim}{ \& }
% List the authors in the Bibilipgraphy as Lastname F, Familyname G,
\DeclareNameAlias{sortname}{family-given}
% remove the punctuation between author names in Bibliography 
%\renewcommand{\revsdnamepunct}{ }


%% Block of definitions for fonts and packages for picture management.
%% In some systems, the figure packages may not be happy together.
%% Choose the ones you need.

%\usepackage[utf8]{inputenc} % For UTF8 support, in some systems. Use UTF8 when saving your file.

\usepackage{lmodern}         % Font package, again in some systems.
\usepackage{textcomp}        % Package for special symbols
\usepackage[pdftex]{color, graphicx} % For pdf output and jpg/png graphics
\usepackage{epsfig}
\usepackage{subfigure}
\usepackage[pdftex, plainpages=false]{hyperref} % For hyperlinks and pdf metadata
\usepackage{fancyhdr}        % For nicer page headers
\usepackage{tikz}            % For making vector graphics (hard to learn but powerful)
%\usepackage{wrapfig}        % For nice text-wrapping figures (use at own discretion)
\usepackage{amsmath, amssymb} % For better math
\usepackage{verbatim}
\usepackage{minted}
\usemintedstyle{vs}

\singlespacing               %line spacing options; normally use single

\fussy
%\sloppy                      % sloppy and fussy commands can be used to avoid overlong text lines
% if you want to see which lines are too long or have too little stuff, comment out the following lines
% \overfullrule=1mm
% to see more info in the detailed log about under/overfull boxes...
% \showboxbreadth=50 
% \showboxdepth=50



%%%%%%%%%%%%%%%%%%%%%%%%%%%%%%%%%%%%%%%%%%%%%%%%%%%%%%%%%
%% STEP 2:
%%%%%%%%%%%%%%%%%%%%%%%%%%%%%%%%%%%%%%%%%%%%%%%%%%%%%%%%%
%% Set up personal information for the title page and the abstract form.
%% Replace parameters with your information.
\title{Behandling av massiva dataströmmar}

% TM: Contributors to template editors now listed in the beginning of the file in comments
\author{Hugo Holmqvist}
\date{\today}



% Set supervisors and examiners, use the titles according to the thesis language
% Prof. 
% Dr. or in Finnish toht. or tri or FT, TkT, Ph.D. or in Swedish... 
\supervisors{Jeremias Berg}
\examiners{Patrik Floréen}


\keywords{dataströmmar, Big Data, lambda-arkitektur, kappa-arkitektur}
\additionalinformation{\translate{\track}}

%% For seminar reports:
%%\additionalinformation{Name of the seminar}

%% Replace classification terms with the ones that match your work. ACM
%% ACM Digital library provides a taxonomy and a tool for classification
%% in computer science. Use 1-3 paths, and use right arrows between the
%% about three levels in the path; each path requires a new line.

\classification{\protect{\ \\
\  Computer systems organization $\rightarrow$ Real-time system architecture  \\
\  Information systems $\rightarrow$ Data stream mining
}}


%% if you want to quote someone special. You can comment this line out and there will be nothing on the document.
%\quoting{Bachelor's degrees make pretty good placemats if you get them laminated.}{Jeph Jacques}


%% OPTIONAL STEP: Set up properties and metadata for the pdf file that pdfLaTeX makes.
%% Your name, work title, and keywords are recommended.
\hypersetup{
    unicode=true,           % to show non-Latin characters in Acrobat’s bookmarks
    pdftoolbar=true,        % show Acrobat’s toolbar?
    pdfmenubar=true,        % show Acrobat’s menu?
    pdffitwindow=false,     % window fit to page when opened
    pdfstartview={FitH},    % fits the width of the page to the window
    pdftitle={Behandling av massiva dataströmmar},            % title
    pdfauthor={Hugo Holmqvist},           % author
    pdfsubject={},          % subject of the document
    pdfcreator={Hugo Holmqvist},          % creator of the document
    pdfproducer={pdfLaTeX}, % producer of the document
    pdfkeywords={something} {something else}, % list of keywords for
    pdfnewwindow=true,      % links in new window
    colorlinks=true,        % false: boxed links; true: colored links
    linkcolor=black,        % color of internal links
    citecolor=black,        % color of links to bibliography
    filecolor=magenta,      % color of file links
    urlcolor=cyan           % color of external links
}

%%-----------------------------------------------------------------------------------



\begin{document}


% Generate title page.
\maketitle


%%%%%%%%%%%%%%%%%%%%%%%%%%%%%%%%%%%%%%%%%%%%%%%%%%%%%%%%%
%% STEP 3:
%%%%%%%%%%%%%%%%%%%%%%%%%%%%%%%%%%%%%%%%%%%%%%%%%%%%%%%%%
%% Write your abstract to be positioned here.
%% You can make several abstract pages (if you want it in different languages),
%% but you should also then redefine some of the above parameters in the proper
%% language as well, in between the abstract definitions.

\begin{abstract}

I denna avhandling går vi igenom två arkitekturer för behandling av massiva
dataströmmar. Vi går även igenom underliggande koncept som krävs för att
förstå själva avhandlingen och introducerar några platformer som kan
användas.

Vi presenterar nödvändiga komponenter för att bygga ett system som kan
behandla massiva dataströmmar, och bevisar hur dom kan användas med hjälp
av konkreta exempel. I avhandlingen diskuterar vi när Kappa- och 
Lambda-arkitektur kan användas, och i vilka situationer ena är bättre än
andra.

\end{abstract}



% Place ToC
\newpage
\mytableofcontents
\mainmatter

%%%%%%%%%%%%%%%%%%%%%%%%%%%%%%%%%%%%%%%%%%%%%%%%%%%%%%%%%
%% STEP 4: Write the thesis.
%%%%%%%%%%%%%%%%%%%%%%%%%%%%%%%%%%%%%%%%%%%%%%%%%%%%%%%%%
%% Your actual text starts here. You shouldn't mess with the code above the line except
%% to change the parameters. Removing the abstract and ToC commands will mess up stuff.
%%
%% You may wish to include material to avoid browsing the definitions
%% above. Command \include{file} includes the file of name file.tex.
%% As a side effect, subsequent inclusions may force a page break.

% BSc instructions
%\include{bsc_finnish_contents}
%\include{bsc_english_contents}
% MSc instructions
%\chapter{Johdanto}

Seuraavassa on joitain ohjeita tämän tutkielmapohjan käyttöön maisterintutkielmassa. Kirjoittamisohjeita löytyy useasta eri lähteestä. Voit esimerkiksi tutustua kandidaatintutkielman ohjeisiin. 
Ohjaajan kanssa on hyvä keskustella aikaisessa vaiheessa työn rakenteesta.

\chapter{Kuvat ja Taulukot}

\section{Kuvat}
Kuva~\ref{fig:logo} toimii esimerkkinä kuvan lisäämisestä työhön. Muista myös viitata jokaiseen kuvaan tekstissä. 

\begin{figure}[h!] % remove [h!] for automatic placement, which is probalby better for a thesis with more text on page
\centering 
\includegraphics[width=0.3\textwidth]{HY-logo-ml.png}
\caption{Helsingin yliopiston logo matemaattis-luonnontieteellisen tiedekunnan värein.\label{fig:logo}}
\end{figure}

\section{Taulukot}

Taulukossa~\ref{table:results} on esimerkki kokeellisten tulosten raportoinnista taulukkona. Muista myös viitata jokaiseen taulukkoon tekstissä.

\begin{table}[h!]
\centering
\caption{Kokeelliset tulokset.\label{table:results}}
\begin{tabular}{l||l c r} 
Koe & 1 & 2 & 3 \\ 
\hline \hline 
$A$ & 2.5 & 4.7 & -11 \\
$B$ & 8.0 & -3.7 & 12.6 \\
$A+B$ & 10.5 & 1.0 & 1.6 \\
\hline
%
\end{tabular}
\end{table}

\chapter{Viitteet}

\section{Kirjallisuusviitteet}

Kirjallisuuslähteet ylläpidetään erillisessä .bib-tiedostossa. Tässä tutkielmapohjassa käytetyt kirjallisuuslähteet, joista esimerkkejä kuvassa~\ref{bibexamples}, löytyvät tiedostosta\newline \texttt{bibliography.bib}.

\begin{figure}[h!]
    \centering
    \begin{scriptsize}
				
\begin{verbatim}

@article{einstein,
    author =       "Albert Einstein",
    title =        "{Zur Elektrodynamik bewegter K{\"o}rper}. ({German})
        [{On} the electrodynamics of moving bodies]",
    journal =      "Annalen der Physik",
    volume =       "322",
    number =       "10",
    pages =        "891--921",
    year =         "1905",
    DOI =          "http://dx.doi.org/10.1002/andp.19053221004"
}
@book{latexcompanion,
    author    = "Michel Goossens and Frank Mittelbach and Alexander Samarin",
    title     = "The \LaTeX\ Companion",
    year      = "1993",
    publisher = "Addison-Wesley",
    address   = "Reading, Massachusetts"
}
@book{knuth99,
    author    = "Donald E. Knuth",
    title     = "Digital Typography",
    year      = "1999",
    publisher = "The Center for the Study of Language and Information",
    series    = "CLSI Lecture Notes (78)"
}
\end{verbatim}
\end{scriptsize}
    \caption{Esimerkkejä kirjallisuuslähteiden kuvaamisesta .bib-tiedostossa.}
    \label{bibexamples}
\end{figure}

Viitteet kirjallisuuslähteisiin muodostetaan komennolla \texttt{\textbackslash citep\{einstein\}}, josta generoituu tekstiin valitun viittaustyylin mukaisesti muotoiltu viite \citep{einstein}, tai \texttt{\textbackslash citep\{latexcompanion,knuth99\}}, josta tekstiin puolestaan generoituu \citep{latexcompanion,knuth99}. Tekstissä viitatut kirjallisuuslähteet tulevat automaattisesti viiteluetteloon. Kirjallisuuslähteiden tietojen oikeellisuus ja yhdenmukaisuus .bib-tiedostossa vaikuttavat luonnollisesti siihen, miten tiedot tutkielmassa näyttäytyvät. Tämä on syytä huomioida, sillä esim. verkosta valmiiksi {Bib\TeX} muodossa löytyvien tietojen täydellisyyten tai samanmuotoisuuteen ei pidä sokeasti luottaa.  

Keskustele viittaustyylin valinnasta ohjaajan kanssa. Joitain vaihtoehtoja on osoitteessa:\\ 
\url{https://www.overleaf.com/learn/latex/Biblatex_bibliography_styles}.
%\url{https://www.sharelatex.com/learn/Bibtex_bibliography_styles}.

\section{Ristiviitteet}

%Liite~\ref{appendix:model} sivulla~\pageref{appendix:model} sisältää lisämateriaalia.
Taulukossa~\ref{table:results} sivulla~\pageref{table:results} on koottuna kokeelliset tulokset.

% \chapter{Pdf:n luominen tex:stä}

% Linuxissa voit ajaa komentoja \texttt{pdflatex filename.tex} ja \texttt{biber filename.tex} vuorotelleen kunnes ohjelmat eivät enää anna varoituksia. Prosessi on helppo automatisoida make-komennon avulla.
 
% \chapter{Johtopäätökset\label{chapter:conclusions}}

% Tutkielma on hyvä päättää johtopäätöksiin tutkielman löydöksistä. 

\chapter{Inledning}

Mängden av producerad data ökar helta tiden, och kräver därför kraftigare och pålitligare
metoder för att hantera datan. Ett exempel på helheter som skapar mycket data i form av
dataströmmar är sakernas internet (Internet of Things, IoT) system som smarta städer
(eng. Smart Cities) var data amlas från sensorer och mobiltelefoner \citep{vakali2014smart}
eller i sjöfart var data samlas från skepp \citep{xu2019internet}. Utvinnande och analysering
av massiva dataströmmar har blivit mera aktuellt eftersom nätverksinfrastrukturen förbättras.

Inkommande datan är kan oftast inte direkt sparas eller analyseras. Därför är det normalt att
segmentera, filtrera eller grupera datan före själva analyseringen utförs på datan.
Datan kan också komma från många olika källor \citep{beringer2006online} som kräver sammanslagning
av datan från källorna. Eftersom mängden data och behovet för analysering av realtids data växer
är behovet för komplexare arkitekturer för databehandling större. Denna texten fokuserar på metoder för
bearbetning av massiva dataströmmar på arkitekturnivå. Vi går även 
igenom några platformer som används för distribuerad bearbetning av dataströmmar.

Det finns två olika sätt för behandling av dataströmmar: Lambda arkitektur 
($\lambda$-arkitektur) och Kappa arkitektur ($\kappa$-arkitektur). Det visar sig att
båda akitekturerna har nackdelar, och att beroende på vilken man väljer måste man
offra performans eller komplexitet \citep{mci/Feick2018}.

Avhandlingen beskriver och jämför både $\lambda$ och $\kappa$ arkitekturen. Bakgrundskunskaperna
för båda arkitekturerna introduceras i kapitel 2 och kapitel 3 går 
igenom några platformer för distribuerad bearbetning. I kapitel 5 diskuteras tilfällen var $\kappa$ 
eller $\lambda$ arkitektur kan användas.

\section{Använda betäckningar}

I texten används olika betäckningar för att beskriva datastrukturer och deras operationer. Betäckningarna är beskrivna i tabellen nedan.

\begin{tabular}{ |p{3cm}||p{8cm}|  }
 \hline
 \multicolumn{2}{|c|}{Betäckningar} \\
 \hline
 Betäckning & Beskrivning\\
 \hline
  $A$[]   &  En lista med element av typ A \\
 $S$[$a$]   &  Läsoperationen av en lista eller dataström $S$. $a$ är indexet för läsoperationen \\
 $(A, B)$   &  En tupel var vänstra värdet är av typ A och högra värdet är av typ B\\
 $S$ & Teckensekvens (eng. string)\\
 \hline
\end{tabular}



\chapter{Introduktion}

Detta kapitel går igenom gruderna för dataströmmar och olika behandligsmetoder. Kapitlet antar att
läsaren kan grunderna i programmering, algoritmer, datanätverk och matematik på kandidatnivå i datavetenskap.

\section{Dataströmmar}

En dataström kan tänkas vara en lista $S$ med $n$ element. Elementen kans läsas från listan med operationen $S$[$t$] 
var $t$ är tidsstämpeln av nutiden. Värden kan inte läsas med framtida eller förflutna tidsstämplar. Listan läses
så ofta som möjligt och evetuella värden behandlas sedan. Eftersom datan är i form av en ström kan vissa 
tidsstämplar ge odefinierade värden, som ignoreras.

\begin{verbatim}
    while (value = stream.readNext()) {
      if (isDefined(value)) {
        processStreamValue(value)
      }
    }
\end{verbatim}

Datan från dataströmmar kan levereras på många olika sätt. Inom IoT är det normalt att dataströmmar 
levereras över TCP/IP \citep{shang2016challenges}. Dataströmmen kan till exempel behandlas med mjukvara som körs på en server eller ett kluster av många servrar.

Bild, mera källor, bättre matematisk beskrivning?

\section{Intagning av dataströmmar}

Intagning (eng. ingestion) av dataströmmar är prosessen var datan hämtas eller mottags från en extern källa.
En platform som används ofta för intagning av data Apache Kafka. Kafka är en platform som fungerar med
en producent/konsument modell (eng. producer/consumer model) [Kafka]. I praktiken betyder det att inkommande data läses av konsumenter och produceras av producenter [Fig1].

\begin{figure}[h]
    \centering
    \includegraphics[scale=0.7]{src/thesis/img/prod-cons-model.png}
    \caption{Producent/konsument modellen med ett Kafka kluster.}
    \label{fig:mesh1}
\end{figure}

Datan som produceras av producenter i Kafka är händelser (eng. events). Händelserna har 3 olika
element: ett nyckelvärde, själva värdet av händelsen och en tidsstämpel. Till exempel, om vi har ett Kafka kluster som intar telemetri från fartyg på havet så kan nyckelvärdet vara ett unikt nummer som fartyget har i ett datasystem så att händelsen kan kopplas samaan med information av fartyget, värdet är kursen av fartyget och tidsstämpeln tiden när fartyget har skickat händelsen.

Kafka ger också möjligheten att dela in händelser i olika kategorier tom kallas ämnen (eng. topics). Konsumenterna prenumererar till olika ämnen och får händelser från prenumererade ämnen.

\section{MapReduce}

När vi har förmågan att inta data från en dataström behöver vi också ett sätt för att hantera, slå ihop och analysera datan som kommer in. En mycket använd model är MapReduce modellen. MapReduce är en programmeringsmodel som kan bearbeta massiva mängder data på ett kluster av datorer \citep{dean2008mapreduce}. MapReduce baserar sig på två funktioner 
som är bekanta från funktionella programmerings paradigmen, \textit{map} och \textit{reduce}. Data splittras först i mindre bitar och bitarna ges till MapReduce \textit{arbetare} (workers) som körs på en annan maskin eller process från huvudprocessen. Varje arbetare
kan få en eller flera bitar av datan. Arbetarna kör först funktionen map på datan och sedan reduce.

\subsection{Map och Reduce funktionerna}

MapReduce modellens map funktion kan beskrivas som en funktion $f: A \rightarrow B$ var $A = (S, S)$ och $B = (S, K)$ var $K$ är typen av datan som returneras. Map funktionen kan användas för till exemple filtrering,
transformering eller tolkning av serialiserad data. Funktionen returnerar en lista av tuplar som innehåller nyckeln och värdet. Map funktionen kans se ut till exempel såhär:

\begin{verbatim}
    def map(key: S, persons: S[]): (S, S)[] = {
      groupedByNameLength = persons.groupByNameLength()
      result = []
      for len in groupedByNameLength:
        append (len, 1) to result
      return result
    }
\end{verbatim}

MapReduce modellens reduce funktion tar in datan som returneras från map funktionen och slår ihop datan. 

\begin{verbatim}
    def reduce(keyFromMap: S, valueFromMap: (Int, Int)[]): Integer = {
      result = 0
      for value in valueFromMap:
        result = result + value
      return result
    }
\end{verbatim}

\subsection{MapReduce processen}

Ett MapReduce program har 7 olika steg. Först splittrats inkommande datan till $N$ segment. Efter splittringen
görs det flera nodar av MapReduce programmet var en nod fungerar som \textit{mästare} (master) och andra
som arbetare. Mästarnoden tilldelar arbetarna map eller reduce arbeten. Arbetarna som har som uppgit att utföra map arbeten
läsen ett segment som mästarnoden har givit den, kör funktionen map och sparar tupel listan i en buffert i minnet.
Innehållet av bufferterna skrivs under jämna mellanrun till en fil och arbetaren signalerar mästarnoden var datan för ett visst
tupel finns i MapReduce klustret. Mästarnoden informerar arbetaren med reduce uppgifter om platser var datan sparad från map skedet
befinner. Datan läses och sorteras sedan så att tuplar med samma nyckel är gruperade tillsammans. Arbetaren kör reduce funktionen på varje
unika nyckel som finns is matade datan. Varje arbetare med ett reduce arbete skriver ut resultatet till en fil.

\begin{figure}[h]
    \centering
    \includegraphics[scale=0.5]{src/thesis/img/map-reduce.png}
    \caption{MapReduce processen}
    \label{fig:mesh1}
\end{figure}

\section{Satsvis bearbetning av dataströmmar}

\textit{Satsvis behandling} (batch processing) av dataströmmar är en metod för behandling av 
dataströmmar som går ut på att köra behandlingsprogram på datan till exempel mellan jämna
tidsinterval eller när någon viss tröskel arr överskridit \citep{marz2013big}. Satsvisa behandligen
innehåller 3 olika steg. I först steget sparas inkommande datan. Inkommande datan kans sparas vart
som hällst, men det är normalt at använda till exemple Apache Hadoops \textit{distribuerade 
filsystem} (Hadoop Distributed File System). I nästa steget processeras datan med till exempel
MapReduce som vi gick igenom i förra kapitlet. MapReduce processens data sparas oftast i en 
nyckel/värde (key-value) databas som sedan kan användas för at servera data tille konsumenter.

\begin{figure}[h]
    \centering
    \includegraphics[scale=0.5]{src/thesis/img/batch-pipeline.png}
    \caption{MapReduce processen}
    \label{fig:mesh1}
\end{figure}

\section{Speed behandling}

\chapter{Arkitektur}

\section{Lambda arkitektur}

\section{Kappa arkitektur}

\section{Andra}

\chapter{Sammanfattning}


%%%%%%%%%%%%%%%%%%%%%%%%%%%%%%%%%%%%%%%%%%%%%%%%%%%%%%%%%
\cleardoublepage                          %fixes the position of bibliography in bookmarks
\phantomsection
\addcontentsline{toc}{chapter}{\bibname}  % This lines adds the bibliography to the ToC
\printbibliography

%%%%%%%%%%%%%%%%%%%%%%%%%%%%%%%%%%%%%%%%%%%%%%%%%%%%%%%%%
\backmatter

%%%%%%%%%%%%%%%%%%%%%%%%%%%%%%%%%%%%%%%%%%%%%%%%%%%%%%%%%

\end{document}
