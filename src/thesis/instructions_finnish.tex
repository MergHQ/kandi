
\appendix{Tutkielmapohjan käyttöohjeet}


HY-CS-main.tex tiedosto sisältää viisi askelta STEPS 0--5. Alla on kuvattu, mitä nämä askeleet tarkoittavat ja miten niitä seuraamalla luot itsellesi oikeanlaisen tutkielmapohjan.
\vspace{0.5cm}

\textbf{STEP 0 -- Kopioi itsellesi tutkielmapohja}

\begin{itemize}
\item Tämä sama pohja on käytössä kaikille tutkielmatyypeille:\\ \url{https://www.overleaf.com/read/hzgngkgshqwh}
\end{itemize}


{\textbf{STEP 1 -- BSc vai MSc tutkielma?}}
\begin{enumerate}
\item Valitse (tiedostossa HY-CS-main.tex) oletko tekemässä BSc (tkt uusi tutkinto, tktl vanhatutkinto) vai MSc (csm uusi tutkinto/30~op, cs vanha tutkinto/40~op)
%, dsm datatiede) 
tutkielman.
\item Valitse kieli jolla kirjoitat tutkielman: finnish, english, tai swedish.
\item Jos olet kirjoittamassa MSc tutkielmaa, valitse linja/opintosuunta.
\end{enumerate}


{\textbf{STEP 2 -- Aseta henkilökohtaiset tietosi}}

\begin{enumerate}
\item Kirjoita alustava otsikko tutkielmallesi.
\item Kirjoita oma nimesi.
\item Kirjoita ohjaajien ja tarkastajien nimet (mikäli tiedossa).
\end{enumerate}

{\textbf{STEP 3 -- Kirjoita tiivistelmä(t)}}

\begin{itemize}
\item Voit kirjoittaa tiivistelmän (koko tiivistelmäsivu) eri kielillä \texttt{otherlanguages}-ympäristön avulla. Alla esimerkki jolla kirjoitat englanninkielisen tiivistelmän muulla kuin englannin kielellä kirjoitettuun tutkielmaan:

\begin{verbatim}
\begin{otherlanguage}{english} 
\begin{abstract}
Your abstract text goes here. 
\end{abstract} 
\end{otherlanguage}
\end{verbatim}

\end{itemize}

{\textbf{STEP 4 -- Kirjoita tutkielma}}

\begin{enumerate}
\item Tutkielman kirjoittamista varten löydät ohjeita tiedostosta \newline \texttt{[bsc/msc]\_[finnish/english]\_contents.tex}.
\item Kun olet tutustunut ohjeisiin, voit poistaa tiedoston \newline \texttt{[bsc/msc]\_[finnish/english]\_contents.tex} sisällön ja kirjoittaa oman tutkielmasi kyseiseen tiedostoon.
\end{enumerate}

{\textbf{STEP 5 -- Aseta kirjallisuuslähdeluettelon tyyli}}

\begin{itemize}
\item Oletustyyli tekijä-vuosi, eli (Einstein, 1905), voit vaihtaa tyylin (tiedostossa HY-CS-main.tex) helposti kommentointia muuttamalla numeroituun, eli [1], tai aakkostyyliin, eli [Ein05].
Lisää ohjeita liittyen viittaustyylin säätämiseen {Bib}\TeX issä löytyy verkosta: \url{https://ctan.org/pkg/biblatex}

\end{itemize}
